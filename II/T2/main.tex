\documentclass{article}
\usepackage{graphicx} 
\usepackage{hyperref}
\hypersetup{
    colorlinks=true,
    linkcolor=blue,
    filecolor=magenta,      
    urlcolor=blue,
    pdftitle={a3},
    pdfpagemode=FullScreen,
    }

\title{Proposta de Tema: Low-Code}
\author{Gabriel Lima nº 51806 | Diogo Matos nº 54466 | Pedro Gomes nº 54554}
\date{\today}

\begin{document}

\maketitle

\section*{Abstract}

\newline Decidimos abordar o tema do Low-Code no nosso trabalho. Esta escolha foi feita devido à percepção de que este assunto não tem recebido a devida atenção no nosso curso, apesar de ser uma tendência crescente e relevante no mercado de trabalho atual. Low-Code é uma abordagem de desenvolvimento de software que visa simplificar e acelarar o processo de criação de aplicacões, permitindo a desenvolvidores e até mesmo usuários sem experiência em programação criarem soluções \cite{2,4,3} por meio de interfaces gráficas de drag-drop e ferramentas visuais, sendo esta uma das razões para termos escolhido este mesmo tema. Com o low-code, as equipas de desenvolvimento podem criar aplicações mais rapidamente, reduzindo o tempo de codificação manual. Isso é possivel graças à abstração de camadas complexas de codificação, permitindo que os desenvolvedores se concentrem mais na lógica do negócio e menos na implementação técnica. Sendo assim o low-code é bastante benéfico no aspeto der reduzir os custos, aumentar a produtividade e capacidade de inovar mais rapidamente \cite{4}. Existem várias plataformas de low-code já desenvolvidas, sendo uma delas o OutSystems, criada por portugueses e estando no mercado desde 2001, sendo esta uma das mais conhecida mundialmente, temos ainda muitas outras tais como Appian. De realçar que com Low-Code é possivel a criação de aplicações, sejam elas móveis ou websites.

\newline Para aplicações móveis, o Low-Code oferece uma maneira simplificada de desenvolver aplicações para dispositivos iOS e Android. As plataformas de Low-Code geralmente vêm com recursos específicos para criar interfaces de usuário responsivas e amigáveis para dispositivos móveis, permitindo que os desenvolvedores criem rapidamente aplicações nativas ou híbridas com funcionalidades avançadas \cite{6}.

\newline No caso de sites web, o Low-Code também oferece uma solução eficiente. As ferramentas visuais e interfaces de drag 'n drop permitem criar rapidamente páginas da web interativas e dinâmicas. Essas plataformas muitas vezes vêm com bibliotecas de componentes pré-construídos e modelos responsivos que simplificam o processo de design e desenvolvimento. Além disso, o Low-Code pode integrar-se facilmente a outras tecnologias web, como BDs e APIs, permitindo a criação de sites complexos e funcionais \cite{6}.

\newline Para não ficar atrás com os dias de hoje, Low-Code já tem integração com chat-bot, algo que hoje em dia não podia faltar no desenvolvimento com produtividade \cite{7,8}.


\newline Ao explorarmos o Low-code neste trabalho, esperamos destacar a sua importância crescente no panorama tecnológico e incentivar uma maior consideração deste conceito no contexto educacional.

\section*{Distribuição do Trabalho}
\newline O Trabalho foi elaborado de maneira equiparada pelos membros do grupo.



\bibliographystyle{plain}
\bibliography{refe}

\end{document}
